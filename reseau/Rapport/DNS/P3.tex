\chapter{Enregistrement DNS}

\section{A quoi sert l'enregistrement du type A ?}
il s'agit du type de base établissant la correspondance entre un nom canonique et une adresse IP. Par ailleurs il peut exister plusieurs enregistrements A, correspondant aux différentes machines du réseau (serveurs). 
\section{Trouver l'adresse IP de $berlioz.elysee.fr$}
PING berlioz.elysee.fr (62.160.71.251) 56(84) bytes of data. 

\section{Trouver le nom et l'adresse du serveur de nom de domaine $elysee.fr$}
Non-authoritative answer: 
elysee.fr	nameserver = berlioz.elysee.fr. 
Authoritative answers can be found from: 
berlioz.elysee.fr	internet address = 84.233.174.57 
berlioz.elysee.fr	internet address = 62.160.71.251 

\section{A quoi sert l'enregistrement du type NS ?}
Il correspond au serveur de noms ayant autorité sur le domaine. 

\section{Quelle est l'autorité administrative de ce domaine ?}
elysee.fr 
	origin = berlioz.elysee.fr 
	mail addr = postmaster.elysee.fr 
	serial = 2010102701 
	refresh = 21600 
	retry = 3600 
	expire = 3600000 
	minimum = 86400 	

\section{A quoi sert l'enregistrement du type SOA ?}
(Start Of Authority) : le champ SOA permet de décrire le serveur de nom ayant autorité sur la zone, ainsi que l'adresse électronique du contact technique (dont le caractère « @ » est remplacé par un point). 

\section{Quel est l'alias de la machine rr.wikimedia.org?}
rr.wikimedia.org	canonical name = rr.esams.wikimedia.org. 

\section{A quoi sert l'enregistrement du type CNAME ?}
(Canonical Name) : il permet de faire correspondre un alias au nom canonique. Il est particulièrement utile pour fournir des noms alternatifs correspondant aux différents services d'une même machine.

\section{Est-ce qu'une machine pourrait avoir plusieurs noms et plusieurs adresses IP ?}
Une machine peut avoir plusieurs nom et plusieurs adresses grâce à l'option PTR et CNAME.


\section{Quel est le nom DNS associé à l'adresse 193.51.208.13 ?}
 	name = dns.inria.fr. 
\section{A quoi sert l'enregistrement du type PTR ?}
un pointeur vers une autre partie de l'espace de noms de domaines. 
\section{Quel est le serveur de courrier du domaine inria.fr ?}
Non-authoritative answer: 
inria.fr	mail exchanger = 10 mail1-smtp-roc.national.inria.fr. 
inria.fr	mail exchanger = 10 mail4-smtp-sop.national.inria.fr.



\section{A quoi sert l'enregistrement du type MX ?}
(Mail eXchange) : correspond au serveur de gestion du courrier. Lorsqu'un utilisateur envoie un courrier électronique à une adresse (utilisateur@domaine), le serveur de courrier sortant interroge le serveur de nom ayant autorité sur le domaine afin d'obtenir l'enregistrement MX. Il peut exister plusieurs MX par domaine, afin de fournir une redondance en cas de panne du serveur de messagerie principal. Ainsi l'enregistrement MX permet de définir une priorité avec une valeur pouvant aller de 0 à 65 535 : 
\section{Trouver les noms et adresses des serveurs de noms du domaine $columbia.edu$.r}
Avec la commande $dig -q columbia.edu$
;; AUTHORITY SECTION:
columbia.edu.		94	IN	NS	ns1.lse.ac.uk.
columbia.edu.		94	IN	NS	dns2.itd.umich.edu.
columbia.edu.		94	IN	NS	adns1.berkeley.edu.
columbia.edu.		94	IN	NS	adns2.berkeley.edu.
columbia.edu.		94	IN	NS	ext-ns1.columbia.edu.

;; ADDITIONAL SECTION:
adns1.berkeley.edu.	97254	IN	A	128.32.136.3
adns2.berkeley.edu.	151906	IN	A	128.32.136.14
ext-ns1.columbia.edu.	890	IN	A	128.59.59.165


