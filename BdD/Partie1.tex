\chapter{Conception de l'entrepôt}
\section{Sujet d'étude}
Le domaine étudié dans ce projet est celui d'un réseau de transport en commun d'une grande ville.
La mise en place et le maintien d'un tel réseau nécessite la connaissances d'une immense quantité d'informations.
Un entrepôt de données OLAP semble tout à fait adapté pour un tel usage puisqu'il permet le stockage et l'accès à l'information d'une très grande base de données.
Le choix du shéma en étoile assura une certaine s'implicité de création et d'utilisation de la base de données.
De plus nos choix de conceptions vont permettre une double utilité à cet entrepôt :

\begin{itemize}
  \item 
    \textbf{Une aide à la décision quant l'aménagement et à l'entretien du réseau.}\\
  En effet, cet entrepôt sera en mesure de renseigner l'organisme responsable du réseau d'un grand nombre de  statistiques et d'informations quant à fréquentation des utilisateurs.
  Il sera par exemple possible d'avoir une idée de quelles zones requièrent le plus d'aménagements par rapport au taux de fréquentations.
  L'organisme aura donc toujours connaissance de l'état d'utilisation de son parc de manière globale.
  \item	
  \textbf{Une génération d'informations précieuses d'un point de vue commerciale.}\\
    Toutes les données necessaires pour établir une stratégie commerciale ( profils clients, zone dynamiques ou non ... ) pourront être fournies facilement par l'entrepôt présenté dans ce rapport.  
    Cet aspect et non négligeable et permet d'obtenir des informations sur les clients sans les importuner par des enquêtes.
\end{itemize}


\section{Tables et dimensions}
L'entrepôt sera composé de 4 tables de dimensions et d'une table de faits : 
\begin{figure}[!h] 
\centering
  \includegraphics[scale=0.60]{ProjetBD.jpeg}
  \caption{Diagramme de classe sujet}
\end{figure} 

La granularité choisie permet de connaître chaques utilisations d'une station (arrêt de tram ou de bus ...).
Ainsi lorsqu'un utilisateur à une station donnée, monte, sors ou reste dans le véhicule, le fait est enregistré dans la table de faits. 
La structure de la base de données peut alors fournir un grand nombre d'informations sur ce fait : à quel trajet il appartient, sur quelle ligne etc...
