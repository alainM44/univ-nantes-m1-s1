\chapter{Analyse experimentale}
\section{Introduction}
Tous nos tests sont effectués en respectant les démarches suivantes : 
\begin{itemize}
\item
Les relevés temporels sont ceux de l'utilisation CPU. Dans un soucis de minimiser les facteurs faussant les relevés (impacts d'autres processus actifs), nous avons essayé d'avoir un minimun de processus actifs pendant les relevés.
\item
Chaques tests a été effectué 3 fois. Le resultat final étant la moyenne de ces trois tests.
\item
Les tests ont pour objectif de confronter nos raisonement lors de l'analyse théorique (cf : )
\item
ROMAIN ROAMIN




\end{itemize}
\section{BLA}

%– une analyse expérimentale de la complexité temporelle des algorithmes présentés. Cette étude com-
%parera en particulier l’impact du choix de telle ou telle structure sur les méthodes implémentées.
%Elle sera réalisée sur des cartes routières variant de quelques dizaines à plusieurs milliers de nœuds
%et établira l’influence des paramètres tels que la densité du graphe, la diversité des décorations,
%etc. Cette analyse sera mise en regard de l’étude théorique et les éventuels écarts observés seront
%expliqués.
BLABLABLABLA



\section{Conclusion}%% methode la plus efficace
BLABLABLABLA

