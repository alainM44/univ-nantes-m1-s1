\chapter{Introduction}

\renewcommand{\labelitemi}{$\bullet$}
\section{Problème à résoudre}
L'objectif de ce projet est de réalisé un Global Positioning System «amélioré». Sans entrer dans les détails, ils sont réalisés à partir d'une modelisation sous forme de graph orienté d'une carte où les posistions sont des noeuds et les routes sont des arcs. L'application d'algorithmes de plus courts chemin sur ces graphs permet alors de répondre au problème concrét du calcul d'iténéraires.
\\
Cependant il est rare, dans la vie réelle, que ce type de problème admette une unique contrainte. L'idée de réaliser un GPS prenant aussi en compte l'aspect touristique d'un lieu ou d'une route donne les opportunités suivantes : 

\begin{itemize}
\item
Une étude approdondie des algorithmes sur les graphs étudiés en cours pour choisir le plus adapté en fonction du problème et de la structure de donnée concrète.
\item
Des manipulations plus complexes de ces algorithmes puisqu'ils prennent plusieurs (et non un) paramètres en compte.
\item
Ces différents choix de combinaisons d'implémentations et d'algorithmes entraînes des calculs variés.
\end{itemize}

Tous ces aspects sont au coeur même du module Structures complexes et algorithmique. La première partie du projet se contentait d'étudier et d'implémenter des structures et des algorithmes complexes. Au terme de l'ultime partie, nous réalisons à partir de ces outils  théoriques une application concrète.


\section{Aplication réalisée}

